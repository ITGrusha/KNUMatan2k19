\documentclass[12pt]{report}
\usepackage[utf8x]{inputenc}
\usepackage[russian]{babel}
\usepackage{amssymb}
\usepackage{amsmath}
\usepackage[makeroom]{cancel}

\title{Конспекти лекцій з математичного аналізу Анікушина А.В. Модуль 2.}
\author{По поводу неточностей и пожеланий пишите в телеграм @bezkorstanislav}
\date{October 2019}

\begin{document}

\maketitle

\begin{center}
\textbf{\LargeГраниця та неперервність функції}
\end{center}

\textit{Означення.} Точка $x_0$ називаться \textbf{граничною} для множини $A$ якщо $$\forall \varepsilon > 0 \ J_{\varepsilon}(x_0) \cap A \neq \O \textrm{, де } J_{\varepsilon}(x_0) = (x_0 - \varepsilon; x_0 + \varepsilon)\setminus \{ x_0 \}$$

\textit{Зауваження.} Якщо $x_0$~--- гранична точка для $A$, то 
$$\forall \varepsilon > 0 \ |J_{\varepsilon} (x_0) \cap A| = \infty$$
Більше того, $x_0$ є граничною для $A \Longleftrightarrow$ існує послідовність $\{ x \}_{n=1}^{+\infty} : x_n \subset \  \subset A, x_i \neq x_0,\ x_n \to x_0$.
$$(\textrm{\textit{Точки дотику}}) = (\textrm{\textit{Граничні точки}}) \cup (\textrm{\textit{Ізольовані точки}})$$
\textbf{Ізользовані точки}~--- це такі точки, які належать множині, але не є граничними.

Якщо $x_0$ не є граничною для $A$, то:
$$\exists \varepsilon > 0 \ J_{\varepsilon} (x_0) \cap A = \O$$

\textit{Приклад:}
$$A = \Bigg\{ \frac{1}{1}, \frac{1}{2}, \frac{1}{3}, \ldots \Bigg\} = \Bigg\{ \frac{1}{n} \Bigg| \  n \in \mathbb{N} \Bigg\}$$
\begin{center}
Єдина гранична точка множини $A$~--- точка $0$.
\end{center}

Сукупність всіх граничних точок множини $A$ називається \textbf{похідною множиною} $A'$.

\vspace{5mm}

Нехай дана функція $f : \mathbb{R} \to \mathbb{R}$ і точка $x_0 \in D'_f \ (\textrm{$x_0$ є граничною точкою $D_f$})$.

\textbf{Означення границі функції за Гейне.} Якщо $\forall \{ x_n \}_{n=1}^{\infty} \  x_n \in D_f,\ x_n \neq x_0,\  x_n \to x_0$ маємо $f(x_n) \to l$, то число $l$ називається границею функції $f$ в точці $x_0$. Позначається так:

$$\lim_{x \to x_0}f(x) = l$$

Той факт, що $x_n \neq x_0$ є дуже важливим. 

\vspace{20mm}

\textit{Приклад}

$$f(x) = \begin{cases}0, &\textrm{Якщо $x \neq 2$}  
\\ 1, &\textrm{Якщо $x = 2$} \end{cases}$$

$$x_n \to 2,\ x_n \neq 2 \Rightarrow f(x_n) = 0 \to 0 \Rightarrow \lim_{x \to 2}f(x) = 0$$

Якби означення було б без умови $x_n \neq x_0$, то $\lim\limits_{x \to 2}f(x) = 0$ або $\lim\limits_{x \to 2}f(x) = 1$. (Тобто границі не було б).

\vspace{3mm}

\textit{Зауваження.} У загальному випадку $\lim\limits_{x \to x_0}f(x)$ ніяким чином не залежить від $f(x_0)$.

\vspace{5mm}

\textit{Означення.} Якщо $\exists \{ x_n \}_{n=1}^{\infty} : x_n \in D_f,\ x_n \to x_0$ та $f(x_n) \to l$, то число $l$ називається частковою границею функції у точці $x_0$.

\textit{Приклад:}
$$f(x) = \sin \frac{1}{x}$$

$$\nexists \lim_{x \to 0} f(x) \textrm{, але існують, наприклад, послідовності:}$$

$$x_n = \frac{1}{\pi n} \to 0,\ \forall n \ f(x_n) = 0  \Rightarrow f(x_n) \to 0$$
$$x_n = \frac{1}{2\pi  n + \frac{\pi}{2}} \to 0,\ \forall n \ f(x_n) = 1  \Rightarrow f(x_n) \to 1$$

Всі числа з інтервалу $[-1;1]$ є частковими границями функції $f(x)$ при $x \to 0$.

\textbf{Означення границі функції за Коші:}

$$\lim_{x \to x_0} f(x) = \alpha \Longleftrightarrow \forall \varepsilon > 0 \ \exists \delta > 0 : \ \forall x \in D_f,\ x, \alpha \in \mathbb{R}$$
$$ 0 < |x - x_0| < \delta \Longrightarrow |f(x) - \alpha| < \varepsilon$$

\textit{Зауваження.} Означення наведене вище працює для дійсного числа $x_0$. Означення для нескінченності у загальному випадку виглядає так:

$$\lim_{x \to \infty} f(x) = \alpha \Longleftrightarrow \forall \varepsilon > 0  \ \exists M > 0 : $$
$$\forall x \ |x| > M \Longrightarrow |f(x) - \alpha| < \varepsilon$$

\textbf{Теорема.} Ознчення границі функції за Коші і за Гейне еквівалентні.

\textbf{Доведення.} Самі знайдете :)

\vspace{5mm}

\textbf{Теорема про арифметичні дії з границями функцій}. Нехай $x_0$ є граничною точкою для $D_f \cap D_g$. $\exists \lim\limits_{x \to x_0}f(x) = \alpha , \ \exists \lim\limits_{x \to x_0}g(x) = \beta$. Тоді:

$$\exists \lim_{x \to x_0}(f(x) \pm g(x)) = \alpha \pm \beta$$
$$\exists \lim_{x \to x_0}(f(x) \cdot g(x)) = \alpha \cdot \beta$$
$$\textrm{Якщо $\beta \neq 0$, то }\exists \lim_{x \to x_0}\frac{f(x)}{g(x)} = \frac{\alpha}{\beta}$$

\textbf{Доведення.} Якщо я хочу обчислити $\lim\limits_{x\to x_0}(f(x) + g(x))$
$$\forall \{x_n \}_{n=1}^{\infty} : x_n \in D_f,\ x_n \to x_0,\ x_n \neq x_0 \  f(x_n) \to \alpha, g(x_n) \to \beta$$.

З теореми про арифметичні дії з послідовностями отримуємо:

$$f(x_n) + g(x_n) \to \alpha + \beta$$.

Інші твердження доводяться за тим же принципом.

\textbf{Теорема про границю композиції}

$$\textrm{Нехай } \lim_{t \to t_0}\varphi(t) = x_0,\ \lim_{x \to x_0}f(x) = y_0,\ t_0 \in D'_{f \circ \varphi}\textrm{, тоді:}$$
$$\lim_{t\to t_0}f(\varphi(t)) = y_0$$

\textbf{Доведення:}
$$\textrm{Розглянемо } \forall \{ t_n \}_{n=1}^{\infty} : t_n \in D_{f \circ \varphi}, t_n \to t_0, t_n \neq t_0$$
$$x_n = \varphi(t_n) \to x_0. \textrm{ При } x_n \to x_0 \ f(x_n) \to y_0, \textrm{ отже:}$$
$$f(\varphi(x_0)) \to y_0$$

\textit{Зауваження.} В умові теореми лектор залишив цікаву, але важкопомітну неточність. За версією Пані Вікторії ця умова  це: існує такий окіл $J_{\varepsilon}(t_0)$, що $\forall t \in (J_{\varepsilon}(t_0) \cap D_{f \circ \varphi})\setminus \{ t_0\}, \varphi ( t ) \neq x_0$.

\vspace{5mm}

\textbf{Означення.} Нехай точка $x_0 \in (D_f \cap (-\infty;x_0))'$, тоді число $\alpha$ називаєтсья \textbf{лівосторонньою границею} (або \textbf{границею зліва}) функції $f$ в точці $x_0$, якщо:

$$\forall \{ x_n \}_{n=1}^{\infty} : x_n \in D_f, x_n \to x_0, x_n < x_0 \Longrightarrow f(x_n) \to \alpha$$

Позначається як $\lim\limits_{x\to x_0-}$ або $\lim\limits_{x \to x_0-0}$ або $f(x_0 - 0)$.

Аналогічно означується \textbf{правостороння границя} функції.

\textit{Приклад:}

$$f(x) = \begin{cases} 0, & \textrm{якщо } x < 2
\\x, & \textrm{якщо x > 2}
\\3, & \textrm{якщо x = 2}\end{cases}$$
$$f(2-0) = 0$$
$$f(2) = 3$$
$$f(2+0) = 2$$

\textbf{Теорема. (Критерій існування границі)}
$$\textrm{Нехай } x_0 \in (D_f \cap (-\infty; x_0))' \textrm{ і } x_0 \in (D_f \cap (x_0; +\infty))'$$
$$\textrm{Тоді } \exists \lim_{x \to x_0}f(x) = \alpha \Longleftrightarrow \exists f(x_0 + 0) = \alpha, \exists f(x_0 - 0) = \alpha$$

\textbf{Доведення.}
\begin{itemize}
    \item $\Longrightarrow$. Якщо $\lim\limits_{x \to x_0}f(x) = \alpha$, то для $\forall \{ x_n \}_{n=1}^{\infty} : x_n \in D_f, x_n \to x_0, x_n \neq x_0 \  f(x_n) \to \alpha$, тож для довільної послідовності $ \{ x_n \}_{n=1}^{\infty} : x_n \in D_f, x_n \to x_0, x_n < x_0 \  f(x_n) \to \alpha$, тому $f(x_0 - 0) = \alpha$. Для правосторонньої границі аналогічно.
    \item $\Longleftarrow$. Нехай тепер $\{ x_n \}_{n=1}^{\infty}$--- довільна послідовність, така, що $x_n \in D_f, x_n \to x_0, x_n \neq x_0$. Розіб'ємо її на дві підпослідовності $\{ x_{n_k} \}_{k=1}^{\infty}$ та $\{ x_{m_k} \}_{k=1}^{\infty}$, такі, що $\forall k \in \mathbb{N} \ x_{n_k} > x_0, x_{m_k} < x_0$.
    $$f(x_{n_k}) \to \alpha, f(x_{m_k}) \to \alpha \Longrightarrow f(x_n) \to \alpha$$
\end{itemize}

\textbf{Означення.} Функція $f$ задовольняє умову Коші в точці $x_0$, якщо:
$$\forall \varepsilon > 0 \ \exists \delta > 0 : \forall x_1, x_2 \in D_f$$
$$\begin{cases}
   0 < |x_1 - x_0| < \delta \\ 0 < |x_2 - x_0| < \delta
   \end{cases} \Longrightarrow |f(x_1) - f(x_2)| < \varepsilon$$

\vspace{5mm}

\textbf{Теорема.} Функція $f$ має границю в точці $x_0 \in D'_f$ тоді й тільки тоді, коли $f$ задовольняє умову Коші в точці $x_0$.


\textbf{Доведення.} Залишили на самоопрацювання :(

\vspace{5mm}

Нехай $f$ і $g$ --- деякі функції. $D_f = D_g, x_0 \in D'_f$.

\begin{enumerate}
    \item $f = O(g)$, якщо $\exists I_{\varepsilon}(x_0)$ (епсилон-окіл точки $x_0$) і $\exists M > 0$ такі, що $\forall x \in I_{\varepsilon}(x_0) \ |f(x)| \leq M \cdot |g(x)|$.
    \item $f$ і $g$--- функції одного порядку, якщо $f = O(g)$ і $g = O(f)$.
    \item $f = o(g)$, якщо $\forall M > 0 \ \exists I_{\varepsilon}(x_0) : \forall x \in I_{\varepsilon}(x_0) \ |f(x)| \leq M \cdot |g(x)|$.
    \item $f \sim g$, якщо $f - g = o(g)$.
\end{enumerate}

\textit{Приклади:}

Нехай $x_0 = 0$.

\begin{itemize}
    \item $f = x^2, g = x^5$. 
    $$|x^2| \leq M\cdot |x^5| \Longleftrightarrow \frac{1}{M} \leq |x^3| \Longrightarrow x^2 \neq O(x^5)$$
    \item $f = x^2, g = 10x^2$.
    $$|x^2| \leq M|10x^2| \Longleftrightarrow \frac{1}{M} \leq 1 \Longrightarrow x^2 = O(10x^2)$$
    $$|10x^2| \leq M|x^2| \Longleftrightarrow \frac{10}{M} \leq 1 \Longrightarrow 10x^2 = O(x^2)$$
    Отже, $x^2$ і $10x^2$--- функції одного порядку.
    \item $f = x^5, g = x^2$.
    $$|x^5| \leq M\cdot |x^2| \Longleftrightarrow |x^3|\leq M \Longrightarrow x^5 = o(x^2)$$
\end{itemize}

\textit{Зауваження.} Такі властивості дуже залежать від обраної точки $x_0$. Наприклад, нехай $f = x^2, g = x^4$. При $x_0 = 0 \ x^4 = o(x^2)$, а при $x_0 = +\infty \ x^2 = o(x^4)$.

\vspace{5mm}

\textit{Корисне зауваження.} 
\begin{itemize}
    \item $\lim\limits_{x \to x_0}\frac{f(x)}{g(x)} = C, C \neq \infty, C \neq 0, \textrm{ то функції $f$ і $g$ одного порядку в точці $x_0$}$
    \item Якщо в деякому околі точки $x_0 \ g(x) \neq 0$ та $f = O(g)$ $\Longleftrightarrow \frac{f(x)}{g(x)}$ є обмеженою.
    \item $f = o(g) \Longleftrightarrow \frac{f}{g} \to 0, x \to x_0$.
    \item $f \sim g \Longleftrightarrow \frac{f}{g} \to 1, x \to x_0$.
\end{itemize}

\textit{Приклад:}

$$\frac{\sin x}{x} \to 0, x \to +\infty \Longrightarrow \sin x = o(x), x \to +\infty$$

\vspace{5mm}

\textit{Означення.} Функція $f$ називається \textbf{обмеженою на множині} $X$, якщо $f(X)$ є обмеженою множиною.

\textit{Означення.} Функція $f$ називається \textbf{обмеженою в точці} $x_0$, якщо $f$ обмежена в деякому околі точки $x_0$.

\textit{Приклад:}

$$\textrm{Функція } f(x) = \frac{1}{x} \textrm{ є обмеженою в усіх точках, окрім $0$.}$$

\textit{Зауваження.} В околі деякої точки $x_0$:
$$f = O(1) \Longleftrightarrow f \textrm{--- обмежена в точці $x_0$}$$
$$f = o(1) \Longleftrightarrow f \to 0, x \to x_0$$

\textbf{Теорема.} Нехай $x_0$ є граничною точкою $D_f = D_g.$

\vspace{3mm}
\setlength{\tabcolsep}{5mm}
\begin{tabular}{c     c}
    $O(f)O(g) = O(f\cdot g)$ &  $c \cdot O(f) = O(f)$\\
    $O(O(f)) = O(f)$& $o(f)O(g) = o(f g)$\\
    $o(f)o(g) = o(f g)$ & $c \cdot o(g) = o(g)$
\end{tabular}

\textbf{Доведення.} Розглянемо доведення $o(f)O(g) = o(fg)$ (інші доводяться аналогічно). 

Припустимо, що $f \neq 0$ та $g \neq 0$ у деякому околі точки $x_0$.

$$\frac{o(f)O(g)}{f g} = \frac{o(f)}{f} \cdot \frac{O(g)}{g}$$
$$\frac{o(f)}{f} \to 0,\ x \to x_0; \frac{O(g)}{g} \textrm{ є обмеженим в деякому околі точки $x_0$}$$

Отже, за теоремою про добуток нескінченно малої та обмеженої отримуємо:

$$\frac{o(f)}{f} \cdot \frac{O(g)}{g} \to 0$$

\textit{Означення.} Якщо $f(x) = g(x) + o(g(x))$ при $x \to x_0$, то $g(x)$ називається \textbf{головною частиною функції} $f$ при $x \to x_0$.

\textit{Приклад:}
$$\lim_{x\to 0}\frac{\sin x}{x} = 1 \Longleftrightarrow \lim_{x\to 0}\Big(\frac{\sin x}{x} - 1\Big) = 0 \Longleftrightarrow \lim_{x\to 0}\frac{\sin (x) - x}{x} = 0 \Longleftrightarrow \sin (x) - x = o(x)$$
$$\sin x = x + o(x)$$
В цьому прикладі $x$ є головною частиною функції $\sin x$ при $x \to 0$.

$$\lim_{x\to 0 }\Big( 1 + x \Big)^{\frac{1}{x}} = e \Longleftrightarrow \lim_{x\to 0} \ln (1 + x)^{\frac{1}{x}} = 1 \Longleftrightarrow \lim_{x\to 0} \frac{\ln (1+x)}{x} = 1 \Longleftrightarrow $$ 
$$\Longleftrightarrow \lim_{x \to 0} \frac{\ln(x+1) - x}{x} = 0 \Longleftrightarrow \ln(x+1) = x + o(x)$$
В цьому прикладі $x$ є головною частиною функції $\ln(x+1)$ при $x \to 0$.

\vspace{5mm}

\textbf{Теорема.} Розглянемо шкалу функцій $x^n, \ x \in \mathbb{N}$ і вважатимемо, що $x \to 0$.

$$x^m = o(x^n), \ m > n \textrm{ (Наприклад, $x^{10} = o(x^2)$)}$$
$$x^m \cdot o(x^n) = o(x^{n+m})$$
$$O(x^m)\cdot o(x^n) = o(x^{n+m})$$
$$o(x^m)\cdot o(x^n) = o(x^{n+m})$$
$$o(x^m) + o(x^n) = o(x^n), \textrm{ якщо $m \geq n$}$$
$$c\cdot o(x^n) = o(x^n)$$
$$o(x^m) = o(x^n), \textrm{ якщо $m \geq n$}$$

\textit{Приклади:}

\vspace{3mm}

Обчислити:

$$(1 + x - 2x^2 + o(x^2))(x - x^2 + o(x^3)) = $$
$$=x - \cancel{x^2} + o(x^3) + \cancel{x^2} - x^3 + x \cdot o(x^3) - 2x^3 + 2x^4 - 2x^2\cdot o(x^3) + x\cdot o(x^2) - x^2 \cdot o(x^2) + o(x^2) \cdot o(x^3) = $$
$$=x + o(x^3) + x \cdot o(x^3) - 3x^3 + 2x^4 - 2x^2\cdot o(x^3) + x\cdot o(x^2) - x^2 \cdot o(x^2) + o(x^2) \cdot o(x^3) = $$
$$=x + o(x^3) + o(x^4) - 3x^3 + 2x^4 - 2o(x^5) + o(x^3) - o(x^4) + o(x^5) = $$
$$=x - 3x^3 + 2x^4 + o(x^3) + o(x^4) + o(x^5) + o(x^3) + o(x^4) + o(x^5) = $$
$$=x - 3x^3 + 2x^4 + o(x^3)$$
$$\textrm{Оскільки $2x^4 = o(x^3)$, то:}$$
$$x - 3x^3 + 2x^4 + o(x^3) = x - 3x^3 + o(x^3) + o(x^3) = x - 3x^3 + o(x^3)$$

\textbf{Основні асимптотичні формули:}

$$\sin x = x - \frac{x^3}{3!} + o(x^3)$$
$$\cos x = 1 - \frac{x^2}{2!} + o(x^2)$$
$$\tg x = x + o(x^2)$$
$$(1+x)^{\alpha} = 1 + \alpha x + \frac{\alpha (\alpha - 1)}{2!}x^2 + o(x^2)$$
$$\ln (1 + x) = x - \frac{x^2}{2} + o(x^2)$$
$$e^x = 1 + \frac{x}{1!} + \frac{x^2}{2!} + o(x^2)$$
$$\arcsin x = x + o(x^2)$$
$$\arctg x = x + o(x^2)$$
$$\arccos x = \frac{\pi}{2} - \arcsin x$$

\textit{Приклади:}

$$1)\ \lim_{x\to 0} \frac{x^2 + 3x^3  +4x^4}{2x^2 + x^5 - x^6 + x^7} = \lim_{x\to 0} \frac{x^2 + o(x^2)}{2x^2 + o(x^2)} = $$
$$ = \lim_{x\to 0} \frac{1 + o(1)}{2 + o(1)} = \frac{1}{2}$$

$$2)\ \lim_{x\to 0} \frac{1 - \cos^{\alpha}x}{(e^x - 1)\sin x} = \lim_{x\to 0} \frac{1 - (1 - \frac{x^2}{2} + o(x^2))^{\alpha}}{(1 + x + o(x)-1)(x - \frac{x^3}{3!} + o(x^3))} = $$
$$= \lim_{x\to 0} \frac{1 - (1 - \frac{x^2}{2} + o(x^2))^{\alpha}}{(x + o(x))(x + o(x))} =\lim_{x\to 0} \frac{1 - (1 + \alpha (-\frac{x^2}{2} + o(x^2)) + o(-\frac{x^2}{2} + o(x^2)))}{(x + o(x))(x + o(x))} =$$
$$=\lim_{x\to 0} \frac{-\alpha (-\frac{x^2}{2} + o(x^2)) + o(x^2)}{x^2 + o(x^2)} = \lim_{x\to 0} \frac{\alpha \frac{x^2}{2} + o(x^2)}{x^2 + o(x^2)} = \frac{\alpha}{2}$$

\vspace{5mm}

\begin{center}
    \textbf{\Large Неперервність функції}
\end{center}

Нехай $f : \mathbb{R} \to \mathbb{R},\ x_0 \in D_f$.

\textbf{Означення неперервності за Гейне.} Функцію $f$ називають неперервною в точці $x_0$, якщо:
$$\forall \{ x_n\}_{n=1}^{\infty} : x_n \in D_f,\ x_n \to x_0 \Longrightarrow f(x_n) \to f(x_0)$$

\textbf{Означення неперервності за Коші.} Функцію $f$ називають неперервною в точці $x_0$, якщо:
$$\forall \varepsilon > 0 \ \exists \delta > 0 : \forall x \in D_f$$
$$|x - x_0| < \delta \Longrightarrow |f(x) - f(x_0)| < \varepsilon$$

\vspace{3mm}

\textbf{Теорема.} Означення за Коші та Гейне еквівалентні.

\textbf{Доведення.} Це ж очевидно $))0))000)$.

\textit{Приклад:}

\begin{enumerate}
    \item $f(x) = x^3,\ x_0 = 2,\ D_f = \mathbb{R}.\ x_n \to 2, f(x_n) = x_{n}^3 \to 2^3 = 8 = f(x_0)$. Отже, функція є неперервною в точці $x_0 = 2$.
    \item $f(x) = \begin{cases} x^3, & x \neq 2  \\ 1, & x = 2\end{cases}$ 
    
    $x_n \to 2,\ x_n \neq 2 \Longrightarrow f(x_n) = x_n^3 \to 8$, але $f(x_0) = 1$, отже функція не є неперервною в точці $x_0$.
    \item $f(x) = \begin{cases} x^3, & x < 2  \\ x, & x \geq 2\end{cases}$
    
    $x_n \to 2,\ x_n < 2 \Longrightarrow f(x_n) = x_n^3 \to 8$ (Це фактично є лівосторонньою границею)
    $x_n \to 2,\ x_n > 2 \Longrightarrow f(x_n) = x_n \to 2$ (Це фактично є правосторонньою границею)
    
    Лівостороння границя не дорівнює значенню функції, отже функція не є неперервною.
    
    \item $f(x) = \frac{1}{x-2}$
    
    $x_n \to 2,\ f(x_n) \to \infty$
    
    Не є неперервною, бо $f(2) \notin \mathbb{R}$ 
\end{enumerate}

Якщо функція $f$ не є неперервною в точці $x_0 \in D_f$, то $f$ називають \textbf{розривною} в точці $x_0$.

Якщо функція $f$ є неперервною в усіх точках деякою множини $S$, то кажуть, що $f$ \textbf{неперервна на множині} $S$. У такому випадку
$$f \in C(S), \textrm{ де $C(S)$ --- множина неперервних на S функцій}$$

\textit{Приклади позначення:}

$$f \in C(\left[ 1;2 \right])$$
$$f \in C( \left( 0;1 \right])$$
$$f \in C(\mathbb{R})$$

\textit{Приклад:}

Функція Діріхле: 

$$f(x) = \begin{cases} 1, & x \in \mathbb{R}  \\ 0, & x \in \mathbb{R}\setminus \mathbb{Q} \textrm{ ($x$ --- ірраціональне)}\end{cases}$$

Нехай $x_0 \in \mathbb{R}$, тоді:

$$\exists \{ p_k\}_{k=1}^{\infty} : p_k \in \mathbb{Q}, p_k \to x_0$$
$$\exists \{ n_k\}_{k=1}^{\infty} : n_k \in \mathbb{R}\setminus \mathbb{Q}, n_k \to x_0$$

$$f(p_k) = 1 \to 1,\ f(n_k) = 0 \to 0$$

Отже, $f$ розривна в точці $x_0$. Також із цього слідує, що функція розривна в усіх точках.

\vspace{3mm}

\textit{Зауваження 1.} Якщо $x_0$ є граничною точкою області визначення ($x_0 \in D_f \cap D'_f$), то поняття неперервності еквіваленте наступному виразу:
$$f \in C(\{ x_0\}) \Longleftrightarrow \lim_{x\to x_0}f(x) = f(x_0)$$

\textit{Зауваження 2.} Якщо $x_0$ --- ізольована точка з $D_f$, то $f \in C(\{ x_0 \})$

\vspace{3mm}

\textit{Домашнє завдання:}

\begin{enumerate}
    \item Придумати приклад фукнції, яка є неперервною рівно в одній точці.
    \item Придумати приклад функції, яка є неперервною рівно в двох точках.
    \item Чи існує функція, яка є неперервною в усіх раціональних точках і розривною в ірраціональних?
    \item Чи існує функція, яка є неперервною в усіх ірраціональних точках та розривною в раціональних?
\end{enumerate}

\vspace{5mm}

\textbf{Теорема про арифметичні дії з неперервними функціями.} Нехай $f, g \in C(\{ x_0 \}).$ Тоді $f + g,\ f-g,\ f\cdot g,$ неперервні в $x_0$. Якщо при цьому $g(x_0) \neq 0$, то  й $\frac{f}{g} \in C(\{ x_0 \})$.

\textbf{Доведення.} Означення за Гейне + теорема про границі послідовностей.

\vspace{5mm}

\textbf{Теорема про неперервність композиції.} Нехай $f \in C(\{ x_0 \}),\ \varphi \in C(\{ t_0\}),\ \varphi(t_0) = x_0$. Тоді $f(\varphi (t)) = \Phi (t)$ є неперервною в точці $x_0$.

\textbf{Доведення.} Якщо:
$$t_n \in D_{\Phi}  = D_{f \circ \varphi},\ t_n \to t_0$$
$$\varphi(t_n) \to \varphi(t_0),\ t_n \to t_0,\ \varphi(t_n) \in D_f$$
$$f(x_n) \to f(x_0),\ x_n \to x_0$$
Отже, $f(\varphi(t_n)) \to f(\varphi(t_0))$, тобто $\Phi(t_n) \to \Phi(t_0)$.

\vspace{5mm}

\textit{Зауваження.} $f \in C(\{ x_0\}) \Longrightarrow x_0 \in D_f$. Якщо функція не є неперервною в точці $x_0$, то вона розривна. $x_0$ може не належати $D_f$, але бути розривною. Але тоді треба, щоб $x_0$ була граничною.

\textit{Наприклад:} $f(x) = \sqrt{x}$ Вона не є розривною в точці $x_0= -2$, бо є вона не є граничою.. 

\textit{Означення.} Нехай $x_0 \in D'_f$ і при цьому $x_0 \notin D_f$. Тоді вважають, що функція $f$ \textbf{має розрив} у точці $x_0$.

Можливі такі ситуації:
\begin{enumerate}
    \item $f(x_0 - 0) = f(x_0 + 0) = f(x_0)$. Тоді $f \in C(\{ x_0\})$.
    \item $f(x_0 - 0) = f(x_0 + 0) \neq f(x_0)$ (або $f(x_0)$ не визначена). Тоді $f$ має \textbf{усувний розрив} у точці $x_0$.
    \item $f(x_0-0) \neq f(x_0+0)$. Тоді $x_0$ має \textbf{розрив типу "стрибок"} у точці $x_0$.
    \item Хоча б одна з границь $f(x_0 + 0),\ f(x_0 - 0)$ не існує або дорівнює $\infty$. Тоді кажуть, що $f$ має \textbf{розрив другого роду} в точці $x_0$.
\end{enumerate}
\textbf{Розриви першого роду} поділяються на усувний і на стрибок.

\begin{center}
    \textbf{\large Властивості елементарних функцій}
\end{center}

\begin{enumerate}
    \item $y = ax + b$
    \item $y = x^n$
    \item $y = a^x$
    \item $y = \log_{a}x$
    \item $\sin,\ \cos,\ \tg,\ \ctg$
    \item $\arcsin,\ \arccos,\ \arctg,\ \arcctg$    
    \item $y = |x|$
\end{enumerate}

\textbf{Теорема.} Усі ці функції є неперервними на своїй області визначення. 

\textbf{Доведення.} Розглянемо доведення тільки для $\sin x$. Треба довести, що $\sin x \in C(\{x_0\})$.

$$|\sin x - \sin y| = \Big|2 \sin \frac{y-x}{2} \cos \frac{y+x}{2}\Big| \leq$$
$$\leq \Big| 2 \sin \frac{y-x}{2} \Big| \leq 2 \Big| \frac{y-x}{2} \Big| = |y - x|$$
$$\textrm{Тоді } |\sin x_n - \sin x_0| \leq |x_n - x_0|$$
$$\textrm{Якщо } x_n \to x_0 \Longrightarrow \sin x_n \to \sin x_0$$

\textit{Приклади:}
\begin{enumerate}
    \item$f(x) = x^x = e^{\ln x^x} = e^{x \ln x} \Longrightarrow x^x \in C(0; +\infty))$
    \item$\arctg (\tg \frac{x^2 + 3}{x+2})$. ОДЗ функції: $x\neq - 2,\ \frac{x^2 + 3}{x+2} \neq \pi k + \frac{\pi}{2}$. Функція є неперервною в усіх точках ОДЗ, але точки $x = - 2,\ \frac{x^2 + 3}{x+2} = \pi k + \frac{\pi}{2}$ є підозрілими.
    
    Для того, щоб дослідити на неперервніть цю функцію треба використати теорему про суперпозицію (У кожній точці шукаємо лівосторонню і правосторонню границю і розглядаємо випадки).
\end{enumerate}

\textit{Приклади використання попередніх теорем:}

\begin{itemize}
    \item Використання теореми про арифметичні дії. 
    
    Дослідити на перерервність функцію $f(x) = \frac{x+1}{x-2} + 5 \cdot \frac{x^2}{x+5}$. 
    
    З теореми про арифметичні дії отримуємо, що $\frac{x+1}{x+2}$ неперервна в усіх точках, окрім $x = -2$, a $5 \cdot \frac{x^2}{x+5}$ неперервна в усіх точках, окрім $x = -5$. А отже й уся функція неперервна в усіх точках, окрім $x=-2,\ x =-5$. Ці точки є підозрілими і їх варто розглядати окремо.
    
    \item Використання теорему про неперервність суперпозиції
    
    Дослідити на неперервність функцію $f(x) = \tg ([x^2])$.
    
    Функція $y=[x^2]$ неперервна в усіх точках, окрім таких, де $x^2$ --- ціле число. Отже, $y = [x^2]$ неперервна в усіх точках, окрім точок виду $x = 0,\pm 1, \pm \sqrt{2},\pm \sqrt{3}, \ldots$. 
    
    Функція $y = \tg ([x^2])$ неперервна в усіх точках, окрім тих, де $[x^2] = \frac{\pi}{2} + \pi k,\ k \in \mathbb{Z}$. (Спойлер: можна довести, що таких точок не існує).
    
    Отже функція є неперервною в усіх точках, окрім $x = x = 0,\pm 1, \pm \sqrt{2},\pm \sqrt{3}, \ldots$. Такі точки є підозрілими і що робити з ними нам пояснять на практиці.
\end{itemize}

\textit{Означення.} Функція $f$ називається \textbf{монотонно зростаючою} на множині $X \subset D_f$, якщо:
$$\forall x_1, x_2 \in X\ x_1 < x_2 \Longrightarrow f(x_1) < f(x_2)$$

Аналогічно означуються \textbf{монотонно спадна}, \textbf{неспадна} та \textbf{незростаюча} функції

\textit{Наприклад:} 

Фунція $y = -x^2 + 3$ є монотонно зростаючою на множині $(-\infty; 0]$.

\vspace{5mm}

\textbf{Теорема про розриви монотонної функції.} Нехай $f$ --- монотонна функція і $x_0 \in (D_f\cap (-\infty; x_0{}))'$. Тоді $\exists f(x_0 - 0)$.

\textbf{Доведення.} Не втрачаючи загальності припустимо, що $f$ --- зростаюча.

Нехай $S = \sup\limits_{x < x_0} f(x)$. Тоді треба довести, що $x_n \in D_f, x_n \to x_{0-} \Longrightarrow f(x_n) \to S$ (Тобто треба довести, що $S = f(x_0 - 0)$). Нехай $\varepsilon > 0$. Тоді оскільки $S$ --- точка дотику множини $\{ f(x), x < x_0\}$, значить $\{ f(x), x < x_0\} \cap (S - \varepsilon; S) \neq \O$,

$$\textrm{Тобто } \exists x^{*} \in D_f, x^{*} < x_0 : f(x^*) \in (S - \varepsilon; S)$$

Розглянемо окіл $I = (x^*; x + \varepsilon)$. Оскільки $x_n \to x_{0-}$, то $\exists N \ \forall n \geq N\ x_n \in I$, тоді $x_n > x^* \Longrightarrow f(x_n) > f(x^*) \longrightarrow f(x_n) \in (S - \varepsilon; S)$.

\textit{Наслідок.} Нехай $f$ --- монотонна і $x_n \in (D_f \cap (-\infty; x_0))' \cap (D_f \cap (x_0; +\infty))'$. Тоді якщо функція має розрив в точці $x_0$, то цей розрив першого роду.

\vspace{5mm}

\textit{Означення.} Множина $K \subset \mathbb{R}$ називається \textbf{компактом} або \textbf{компактною множиною}, якщо $\forall \{ x_n\}_{n=1}^{\infty},\ x_n \in K$ існує підпослідовність $\{x_{n_k}\}_{k=1}^{\infty} : x_{n_k} \to x_0 \in K{}$.

\textit{Приклади:}

\begin{itemize}
    \item $K = (0; +\infty)$ не є компактом, бо послідовність $x_n = n$ прямує до нескінченності, а отже й будь-яка її підпослідовність буде прямувати до нескінченності.
    \item $K = (0;1)$ не є компактом, бо послідовність $x_n = \frac{1}{n}$ прямує до нуля, але нуль не належить множині $K$.
\end{itemize}

\textbf{Теорема про обмеженість компакту.} Нехай $K \subset \mathbb{R}$ і $K$ --- компакт. Тоді $K$ --- обмежена.

\textbf{Доведення.} Припустимо супротивне, тоді обов'язково існує $\{ x_n\}_{n=1}^{\infty} : x_n \to \infty$. Тоді за означенням компакту в неї має бути збіжна послідовність, але $x_n \to \infty$, а отже й будь-яка її підпослідовність теж прямуватиме до нескінченності, яка не належить множині $K$. \textit{Протиріччя із здоровим глуздом}.

\vspace{5mm}

\textbf{Теорема. (Критерій компакту).} Нехай $K \subset \mathbb{R}$. $K$ --- компакт $\Longleftrightarrow$ $K$ є замкненою і обмеженою.

\textbf{Доведення.} 
\begin{itemize}
    \item $\Longrightarrow$. Обмеженість вже довели. Припустимо, що $K$ не є замкненою, тоді $\exists x_0$ --- точка дотику, яка не належить $K$. Іншими словами $x_0$ --- гранична точка. Оберемо довільну послідовність $\{ x_n\}_{n=1}^{\infty},\ x_n \in K, x_n \to x_0$. Будь-яка її підпослідовність теж збігатиметься до $x_0 \notin K$. Протиріччя.
    \item $\Longleftarrow$. Нехай $\{ x_n\}_{n=1}^{\infty}$ --- довільна  послідовність така, що $x_n \in K$. $x_n \in K$, а $K$ --- обмежена множина, значить сама послідовність $x_n$ є обмеженою.
    $$\textrm{Отже } \exists \{ x_{n_k}\}_{k=1}^{\infty},\ x_{n_k} \in K,\ x_{n_k} \to x_0 \Longrightarrow x_0 \textrm{--- точка дотику}$$
    Але ми знаємо, що множина $K$ є замкненою, а отже містить усі свої точки дотику. Отже, $x_0 \in K$.
\end{itemize}

\textit{Приклади:}
\begin{itemize}
    \item $[0;1)$ --- не компакт.
    \item $(-\infty;0]$ --- не комакт.
    \item $[0;1]$ --- компакт.
    \item $\bigcup\limits_{n=2}^{\infty} \big[\frac{1}{n}; 1 - \frac{1}{n}\big] = (0;1)$ --- не компакт.
\end{itemize}

\textit{Наслідок.} Будь-який компакт має найбільший і найменший елемент.

\vspace{3mm}

\textbf{Математичний жарт.}

Розмова між хлопцем-математиком та дівчиною:
\begin{itemize}
    \item[--] \textit{Ти у мене така компактна!}
    \item[--] \textit{А як це?}
    \item[--] \textit{Замкнена і обмежена.}
\end{itemize}

\vspace{5mm}

\textbf{Теорема про неперервний образ компакта.} Нехай $f \in C(K)$, $K$ --- компакт. Тоді $f(K)$ --- теж компакт.

\textbf{Доведення.} Нехай $\{ y_n\}_{n=1}^{\infty}$ --- довільна послідовність з $f(K)$. Оскільки $\forall n\ \exists x_n \in K : f(x_n) = y_n$, розглянемо послідовність $\{ x_n\}_{n=1}^{\infty}$.  Оскільки $K$ --- компакт, то з $\{ x_n\}_{n=1}^{\infty}$ можна обрати збіжну підпослідовність $\{ x_{n_k}\}_{k=1}^{\infty}$. При чому $x_{n_k} \to x_0 \in \mathbb{R}$.

Оскільки $f$ --- неперервна в точці $x_0$, то $f(x_{n_k}) \to f(x_0)$, а отже $y_{n_k} \to f(x_0) \in f(K)$.

\vspace{3mm}

\textit{Наслідок.} \textbf{(Теорема Вейерштрасса).} Функція, що неперервна на компакті досягає там свого найбільшого та найменшого значень.

\vspace{5mm}

\textbf{Теорема про неперервність оберненої функції.} Нехай $f \in C(\{ K\})$, $K$ --- компакт, а $f$ --- оборотня функція (тобто існує обернена до неї функція). Тоді обернена функція теж є неперервною на $K$.

\textbf{Доведення.} Самі знайдете :) + вам варто зрозуміти чи працює теорема, якщо $K$ не є компактом.

\vspace{5mm}

\textbf{Теорема про монотонність оберненої функції.} Нехай $f$ --- неперервна на компакті і монотонна. Якщо $\exists f^{-1}$ (обернена функція), то $f^{-1}$ теж буде монотонною на цьому компакті.

\vspace{5mm}

\textbf{Теорема Бореля-Лебега.} Нехай $K$ --- компакт і $\{ I_n\}_{n=1}^{\infty}$ --- послідовність інтервалів і при цьому 
$$K \subseteq \bigcup\limits_{n=1}^{\infty} I_n$$ 
Тоді з $I_n$ можна обрати скінченну кількість інтервалів, які будуть покривати $K$.







\end{document} 
