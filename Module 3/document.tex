\documentclass[12pt]{report}
\usepackage[utf8x]{inputenc}
\usepackage[russian]{babel}
\usepackage{amssymb}
\usepackage{amsmath}
\usepackage[makeroom]{cancel}

\title{Конспекти лекцій з математичного аналізу Анікушина А.В. Модуль 3.}
\author{Автор текста @bezkorstanislav \\ Если есть ошибки, пишите ему в телеграм \\  \small{Aфтар выражает благодарность @vic778 за многочисленные поправки}}

\date{October 2019}

\begin{document}

\maketitle

\begin{center}
\textbf{\LargeДиференціальне числення функції однієї змінної}
\end{center}

\begin{center}
\textbf{\large{Означення похідної. Основні правила диференційонування}}
\end{center}

Нехай $f : \mathbb{R} \to \mathbb{R}$ --- деяка функція однієї змінної, $x_0 \in D_f,\ x_0 \in (D_f)^{'}$ .

\textbf{Означення.} Якщо $\exists \lim\limits_{x \to x_0} \frac{f(x) - f(x_0)}{x - x_0}$, то функція $f$ називається 
\textbf{диференційованою} в точці $x_0$, а сама границя називається \textbf{похідною} в точці $x_0$. І позначається $f'(x)$ або $\frac{d f(x)}{d x}$.

\vspace{3mm}

\textit{Зауваження 1.} Для функції однієї змінної ми ототожнили диференційованість та існування похідної.

\textit{Зауваження 2.} $$x - x_0 = \Delta x$$
$$f'(x_0) = \lim_{\Delta x \to 0} \frac{f(x_0 + \Delta x) - f(x_0)}{\Delta x}$$
$$f(x_0 + \Delta x) - f(x_0) = \Delta f(x)$$
Отже, похідна дорівнює відношенню зміни приросту функції до приросту аргументу, що породжує цей приріст.

\textit{Зауваження 3.} 
$$f'(x_0) = \lim_{x \to x_0} \frac{f(x) - f(x_0)}{x - x_0} \Longleftrightarrow \frac{f(x) - f(x_0)}{x - x_0} = f'(x_0) + o(1)$$
$$f(x) - f(x_0) = f'(x_0)(x-x_0) + o(x-x_0)$$
Отже, якщо має місце рівність:
$$f(x) - f(x_0) = A (x - x_0) + o(x - x_0) \Longrightarrow \exists f'(x_0) = A$$

\textbf{Теорема. (Необхідна умова диференційованості).} Функція $f$ є диференційованою в $x_0$ тільки тоді, коли $f$ --- неперервна в точці $x_0$.

\textbf{Доведення.} Щоб існувала похідна треба, щоб
$$f(x) - f(x_0) \to 0,\to x \to x_0$$.
З цього випливає, що $f$ неперервна в точці $x_0$.

\textbf{Теорема. (Диференційованість композиції функцій).}  Нехай дано функції $f$ і $g$. точка $x_0 \in D_{f \circ g},\ x_0 \in (D_{f \circ g})'$.

Якщо $g$ диференційована в точці $x_0$, а $f$ диференційована в точці $y_0 = g(x_0)$, то $f \circ g$ диференційована в точці $x_0$ і має місце рівність:
$$(f \circ g)' (x_0) = f'(g(y_0)) g'(x_0)$$

\textbf{Доведення.} 
$$(f \circ g) (x) - (f \circ g) (x_0) = f(g(x)) - f(g(x_0)) = f(y) - f(y_0) =$$ 
$$= f'(y_0)(y - y_0) + o(y - y_0) = f'(y_0)(g(x) - g(x_0)) + o(y - y_0) = $$
$$= f'(y_0)(g'(x_0)(x - x_0) + o(x - x_0)) + o(y - y_0) = $$
$$= f'(y_0)(g'(x_0)(x - x_0)(1 + o(1))) + o(g(x) - g(x_0)) = $$
$$= f'(y_0)(g'(x_0)(x - x_0)(1 + o(1))) + o(g(x_0)(x - x_0) + o(x - x_0))$$

Отже,

$$\lim_{x \to x_0} \frac{(f \circ g)(x) - (f \circ g)(x_0)}{x - x_0} = $$
$$=\lim_{x \to x_0}\frac{f'(y_0)(g'(x_0)(x - x_0)(1 + o(1))) + o(g(x_0)(x - x_0) + o(x - x_0))}{x - x_0} = $$
$$=\lim_{x \to x_0}f'(y_0)(g'(x_0)(1 + o(1))) + o(g(x_0) + o(1)) = $$
$$=\lim_{x \to x_0}f'(y_0)g'(x_0)(1 + o(1)) + o(1) = f'(y_0)g'(x_0)$$

\textbf{Теорема. (Лінійність похідної).} Нехай $f$ і $g$ --- диференційовані в точці $x_0$, $\alpha,\beta \in \mathbb{R}$, то:
$$(\alpha f + \beta g)' (x_0) = \alpha f'(x_0) + \beta g'(x_0)$$

\textbf{Доведення.} 

$$\lim_{x \to x_0} \frac{(\alpha f + \beta g)(x) - (\alpha f + \beta g)(x_0)}{x - x_0} = $$
$$\lim_{x \to x_0} \frac{ \alpha f(x) - \alpha f(x_0)}{x - x_0} + \frac{\beta g(x) - \beta g (x_0)}{x - x_0} = $$
$$\lim_{x \to x_0} \alpha \frac{ f(x) - f(x_0)}{x - x_0} + \lim_{x \to x_0}\beta \frac{g(x) - g (x_0)}{x - x_0}  = \alpha f'(x_0) + \beta g'(x_0)$$

\textbf{Теорема. Похідна добутку.} Нехай $f$ і $g$ диференційовані в точці $x_0$. 

Тоді функція $f \cdot g$ теж диференційована в точці $x_0$, при чому справедливе наступне співвідношення:
$$(f \circ g)' (x_0) = f'(x_0)g(x_0) + f(x_0) g'(x_0)$$

\textbf{Доведення.} За означенням:

$$\lim_{x \to x_0} \frac{f(x)g(x) - f(x_0)g(x_0)}{x - x_0} = $$
$$= \lim_{x \to x_0} \frac{f(x)g(x) - f(x)g(x_0) + f(x)g(x_0) - f(x_0)g(x_0)}{x - x_0} = $$
$$= \lim_{x \to x_0} \frac{f(x)(g(x) - g(x_0))}{x - x_0} + \frac{g(x_0)(f(x) - f(x_0))}{x - x_0} = $$

Оскільки $f$ --- диференційована в точці $x_0$, то $\exists \lim\limits_{x \to x_0} \frac{f(x) - f(x_0)}{x - x_0} = f'(x_0)$.

Абсолютно аналогічно $\exists \lim\limits_{x \to x_0} \frac{g(x) - g(x_0)}{x - x_0} = g'(x_0)$.

Окрім того $f$ --- диференційована в точці $x_0$, а значить $f$ --- неперервна в точці $x_0$, а значить:
$$\lim_{x \to x_0} f(x) = f(x_0)$$

Отже:

$$\lim_{x \to x_0} \frac{f(x)(g(x) - g(x_0))}{x - x_0} + \frac{g(x_0)(f(x) - f(x_0))}{x - x_0} = $$
$$= f(x_0) g'(x_0) + g(x_0) f'(x_0)$$

\textbf{Теорема. Похідна частки.} Нехай $f$ і $g$ диференційовані в точці $x_0$ і $g(x_0) \neq 0$.

Тоді, $\frac{f}{g}$ теж диференційована в точці $x_0$ і $\Big(\frac{f}{g} \Big)'(x_0) = \frac{f'(x_0) g(x_0) - f(x_0)g'(x_0)}{g^2(x_0)}$

\textbf{Доведення.} Аналогічно. Це доведення на дз :).

\textit{Приклад.} Обчислити похідну від $f$ від функції $f$ в точці $x_0$.
\begin{enumerate}

\item $f(x) = x^3$.
$$f'(x_0) = \lim_{x\to x_0}\frac{f(x) - f(x_0)}{x - x_0} = \lim_{x\to x_0}\frac{x^3 - x_0^3}{x - x_0} = \ldots = 3 x_0^2$$

\item $f(x) = \begin{cases} x^3,& x \neq 2 \\ 8, & x = 2\end{cases}$
$$f'(x_0) = \lim_{x \to x_0}\frac{f(x) - f(x_0)}{x - x_0}$$
\begin{enumerate}
\item $x_0 = 2$. $f'(x_0) = \frac{x^3 - 8}{x - 2} = 3 \cdot 4 = 12.$
\item $x_0 \neq 2$. $\lim\limits_{x \to x_0} \frac{x^3 - x_0^3}{x - x_0} = 3 x_0^2$. 
\end{enumerate}

\item $f(x) = \begin{cases} x^3,& x \neq 2 \\ 7, & x = 2\end{cases}$
$$f'(x_0) = \lim_{x \to x_0}\frac{f(x) - f(x_0)}{x - x_0}$$
\begin{enumerate}
\item $x_0 \neq 2$. $\lim\limits_{x \to x_0} \frac{x^3 - x_0^3}{x - x_0} = 3 x_0^2$. 
\item $x_0 = 2$. $f'(x_0) = \lim\limits_{x\to 2}\frac{x^3 - 7}{x - 2} = \lim\limits_{x\to 2}\frac{8 - 7}{2 - 2} = \lim\limits_{x\to 2}\frac{1}{0} = +\infty$.

Похідна не може дорівнювати нескінченності, отже $\nexists f'(2)$. 
\end{enumerate}

\item $f(x) = \begin{cases} x^3,& x \geq 2 \\ 12x - 16, & x < 2\end{cases}$

Як \textbf{не треба} робити:
$$f'(x) = \begin{cases} 3 x^2,& x \geq 2 \\ 12, & x < 2\end{cases}$$
Як \textbf{треба} робити:

\vspace{1mm}

Розглядаємо $3$ випадки:

\begin{enumerate}
\item $x_0 > 2$.

$$\lim_{x \to x_0}\frac{f(x) - f(x_0)}{x - x_0} = \lim_{x \to x_0}\frac{x^3 - x_0^3}{x - x_0} = 2 x_0^2$$

\item $x_0 < 2$.

$$\lim_{x \to x_0}\frac{f(x) - f(x_0)}{x - x_0} = 12$$

\item $x = 2$.

$$\lim_{x \to x_0}\frac{f(x) - f(2)}{x - 2} = \lim_{x \to 2+0}\frac{f(x) - f(2)}{x - 2} = \lim_{x \to 2-0}\frac{f(x) - f(2)}{x - 2} = 12$$
\end{enumerate}

Отже, функція диференційована на всіх дійсній прямій.
 
\end{enumerate}

\textbf{Теорема. (Про диференційованість оберненої функції)}. Формулювання і доведення на додому.

\textit{Зауваженн.} Функція $f$ називається диференційованою на множині $A$ якщо $f$ має 
похідну в кожній точці $x_0 \in A$. Це позначається як $f \in D(A)$, де $D(A)$ --- множина диференційованих на $A$ функцій. 

\begin{center}
\textbf{\largeОдносторонні похідні}
\end{center}

Нехай $x_0 \in (D_f \cap (-\infty, x_0))' \cap (D_f \cap (x_0, +\infty))'\cap D_f$. Це рівносильно тому, що 
$$\exists \{ x_n\}_{n=1}^{\infty} \in D_f : x_n < x_0,\ x_n \to x_0, n \to \infty $$
Границя $\lim\limits_{x\to x_0-}\frac{f(x) - f(x_0)}{x - x_0}$ називається лівосторонньою похідною $f$ в точці $x_0$. Позначаємо її як $f'_{л}(x)$.

Аналогічно, $f'_{п} (x_0) = \lim\limits_{x \to x_0+} \frac{f(x) - f(x_0)}{x - x_0}$

\vspace{3mm}

\textit{Приклади:}

\begin{enumerate}

\item $f(x) = \begin{cases} 0, & x \leq 0 \\ 1, & x > 1\end{cases}$
$$f'_{л}(x_0) = \lim_{x \to x_0} \frac{0 - 0}{x - 0} = 0$$
$$f'_{п}(x_0) = \lim_{x \to x_0} \frac{1 - 0}{x - 0} = +\infty \Longrightarrow \nexists f'_{n}(x_0)$$

\item $f(x) = \begin{cases} 0, & x \leq 0 \\ x, & x > 1\end{cases}$
$$f'_{л}(x_0) = \lim_{x \to x_0} \frac{0 - 0}{x - 0} = 0$$
$$f'_{п}(x_0) = \lim_{x \to x_0} \frac{x - 0}{x - 0} = 1$$
\end{enumerate}

\textbf{Теорема. (Критерій диференційованості}.) 

Нехай $x_0 \in (D_f \cap (-\infty, x_0))' \cap (D_f \cap (x_0, +\infty))' \cap x_0 \in D_f$. Тоді:
$$\exists f'(x_0) \Longleftrightarrow \exists f'_{л} \exists f'_{п}, f'_{л} = f'_{п}$$

\textbf{Доведення.} 
$$\exists f'(x_0) \Longrightarrow \lim_{x \to x_0}\frac{f(x) - f(x_0)}{x - x_0} \Longleftrightarrow$$
$$\Longleftrightarrow \exists \lim_{x\to x_0+}\frac{f(x) - f(x_0)}{x - x_0},\ \exists \lim_{x\to x_0-}\frac{f(x) - f(x_0)}{x - x_0}, \lim_{x\to x_0+}\frac{f(x) - f(x_0)}{x - x_0} = \lim_{x\to x_0-}\frac{f(x) - f(x_0)}{x - x_0}$$
$$\Longleftrightarrow \exists f'_{п}(x_0),\ \exists f'_{л}(x_0),\ f'_{п}(x_0) = f'_{л}(x_0)$$


\textit{Зауваження.}
$$f'_{л} \neq \lim_{x\to x_0-}f'(x)$$

\vspace{5mm}

Розглянемо параметрично задану функцію $y(x)$:
$$\begin{cases} y = \varphi(t) \\ x = \Psi(t) \end{cases}$$
$$y(x) = \varphi( \Psi^{-1} (x))$$
Часом виразити $y$ через $x$ може бути просто, як тут:
$$\begin{cases} y = t^3 \\ x = t^5\end{cases} \Longleftrightarrow y = x^{\frac{3}{5}}$$
Але буває, що не дуже просто:
$$\begin{cases} y = t^3 \\ x = t^5+t\end{cases} \Longleftrightarrow y = ???$$

Але ж \textbf{так} хочеться знайти похідну...

\vspace{3mm}

Припустимо, що $\exists \varphi'$ і $\exists (\Psi^{-1})'$ і $\Psi' \neq 0$.

$$\textrm{Тоді }y_x' = \varphi' ( \Psi^{-1} (x) ) \cdot (\Psi^{-1})' (x) = \varphi'(t) \frac{1}{\Psi'(t)} = \frac{\varphi'(t)}{\Psi'(t)}$$

\textit{Приклад:}

\begin{enumerate}

\item Знайти похідну $y_x$ 
$$\begin{cases} y = t^3 \\ x = t^5\end{cases} \Longleftrightarrow y = x^{\frac{3}{5}}$$

Ми уже знайшли, що $y = x^{\frac{3}{5}}$, отже:
$$y_x' = \frac{3}{5} x^{-\frac{2}{5}} = \frac{3}{5} x^{-\frac{2}{5}} = \frac{3}{5} (t^5)^{-\frac{2}{5}} = \frac{3}{5} t^{-2}$$ 

Але давайте спробуємо \textit{по-нормальному}, застосувавши нашу формулу:
$$y_x' = \frac{(t^3)'}{(t_5)'} = \frac{3}{5} \frac{t^2}{t^4} = \frac{3}{5} t^{-2}$$

\item $$\begin{cases} y = \cos^2 t \\ x = \sin^2 t \end{cases}$$
$$y_x' = \frac{2 \cos t (- \sin t)}{2 \sin t \cos t} = -1$$
Як цікавий факт можна помітити, що $y = 1 - x$, де $x \in [0;1]$


\end{enumerate}


\begin{center}

\textbf{\large{Похідні вищих порядків}}

\end{center}

Кажуть, що порядок похідної є вищим, якщо цей порядок більше за $1$.

Нехай $f \ in D((a,b))$. Припустимо, що $f'$ є диференційованою в точці $x_0 \in (a,b)$. Тоді функція $f$ є двічі диференційованою в точці $x_0$, а 
число $(f')'(x_0)$ називається \textbf{другою похідною функції} $f$ в точці $x_0$. Вона позначається як $f''(x_0)$.

\vspace{3mm}

\textit{Наприклад:}
$$f(x) = x^3$$
$$f'(x) = 3 x^2$$
$$f''(x) = (f')'(x) = 6x$$

Аналогічно у функції $f$ існує $n-$та  похідна $f^{(n)}(x)$ на проміжку $(a,b)$ і вона є диференційованою в точці $x_0 \in (a,b)$, того
$f$ має $(n+1)$-шу похідну в точці $x_0$.
$$f^{(n+1)}(x_0) = (f^{(n)})'(x_0) = \lim_{x\to x_0}\frac{f^{(n)}(x) - f^{(n)}(x_0)}{x - x_0}$$

\textit{Приклад:}
$$f(x) = \ln x$$
$$f'(x) = \frac{1}{x}$$
$$f''(x) = \Big( \frac{1}{x} \Big)' = -\frac{1}{x^2}$$
$$f'''(x) = \Big( -\frac{1}{x^2} \Big)' = 2\frac{1}{x^3}$$
$$f^{(4)}(x) = -3 \cdot 2 \cdot \frac{1}{x^4} = -6 \frac{1}{x^4}$$
$$f^{(n)}(x) = (-1)^{n-1}(n-1)!\frac{1}{x^n}$$

Якщо $f$ має $n$-ту похідну $f^(n)(x)$ у кожній точці проміжку $I$, то кажуть, що $f$ є $n$ разів диференційованою і позначається це так:
$$f \in D^{(n)}(I)$$

Якщо при цьому $f^{n} \in C(I)$, то кажуть, що $f \in C^{(n)}(I)$, то кажуть, що $f \in C^{(n)}(I)$.

$$C(I) \subset D(I) \subset C^{(1)}(I) \subset D^{(1)}(I) \subset C^{(2)}(I) \subset \ldots$$

\textit{Домашня робота.} Якщо у функції є похідна, чи обов'язково похідна неперервна?

\vspace{3mm}

Якщо $\forall n \in \mathbb{N}\ \exists f^{(n)}$ на $I$, то кажуть, що $f$ --- нескінченно диференційована на $I$. Позначається так:
$$f \in C^{\infty}(I)$$

\begin{center}

\textbf{\large{Дві властивочті $n$-тої похідної}}

\end{center}

\textbf{Теорема. (Лінійність $n$-тої похідної).} Нехай $f,g \in D^{(n)}(I)$, тоді 
$$\forall \alpha, \beta \in \mathbb{R}\ \alpha f  +\beta g \in D^{(n)}(I)$$
$$(\alpha f + \beta g)^{(n)} = \alpha f^{(n)} + \beta g^{(n)}$$

\textbf{Теорема. Формула Лейбніца.} Нехай $f,g \in D^{(n)}(I)$, тоді: 
$$f \cdot g \in D^{(n)}(I)$$
$$(f g)^{(n)} = \sum_{k = 0}^{n} C_n^k \cdot f^{(n-k)} \cdot g^{(k)}$$

\textbf{Доведення.} Це вам нам Д/З. (Підсказка: воно доводиться также само, як формула бінома Ньютона).

\vspace{3mm}

\textit{Приклад:}
\begin{enumerate}
\item
Обчислити $n$-ту похідну від $f(x) = x^2 \ln x$.

$$(x^2 \ln x)^{(n)} = C_n^0 (x^2)^{(n)} \ln x + C_n^1 (x^2)^{(n-1)} (\ln x)^{(1)} + C_n^2 (x^2)^{(n-2)} (\ln x)^{(2)} + \ldots + $$
$$+ \ldots + C_n^{n-3} (x^2)^{(3)} (\ln x)^{(n - 3)} + C_n^{n-2} (x^2)^{(2)} (\ln x)^{(n - 2)} + C_n^{n-1} (x^2)^{(1)} (\ln x)^{(n - 1)} + $$
$$+ C_n^{n} (x^2)^{(2)} (\ln x)^{(n - 2)}$$

Тепер варто помітити, що:

$$(x^2)^{(1)} = 2x$$
$$(x^2)^{(2)} = 2$$
$$(x^2)^{(3)} = 0$$
$$(x^2)^{(n)} = 0,\ n > 3$$

Отже виходить, що наша сума дорівнює наступному:

$$C_n^{n-2}  \frac{2 (-1)^{n-3}(n-3)!}{x^{n-2}} + C_n^{n-1}  \frac{2x (-1)^{n-2}(n-2)!}{x^{n-1}}+ C_n^{n} \frac{ x^2 (-1)^{n-1}(n-1)!}{x^{n}}$$

\item Обсчислити $(\frac{1}{a-x})^{(n)}$

$$(\frac{1}{a-x})^{(0)} = \frac{1}{a-x}$$
$$(\frac{1}{a-x})^{(1)} = \frac{1}{(a-x)^2}$$
$$(\frac{1}{a-x})^{(2)} = \frac{2}{(a-x)^3}$$
$$(\frac{1}{a-x})^{(n} = \frac{n!}{(a-x)^{n+1}}$$

\item Обчислити $n$-ту похідну від $f(x) = \frac{1}{x^2 - 3x + 2}$.

$$f(x) = \frac{1}{x^2 - 3x + 2} = \frac{1}{(x-1)(x-2)} = \frac{1}{x-1} \frac{1}{x-2} = \frac{1}{1-x} \frac{1}{2-x} = $$

З формули Лейбніца:

$$f^{(n)} = \sum_{k = 0}^{n} C_n^k \Big( \frac{1}{1-x}\Big)^{(n-k)} \Big( \frac{1}{2-x}\Big)^{(k)}$$

А тепер використовуємо формулу з прикладу $2$.

$$f^{(n)} = \sum_{k = 0}^{n} C_n^k \Big( \frac{(n-k)!}{(1-x)^{n-k+1}}\Big) \Big( \frac{(k!)}{(2-x)^{k+1}}\Big)$$

Цікавий факт: конкретно у цьому завданні можна було зробити набагато простіше:

$$f(x) = \frac{1}{1-x} \cdot \frac{1}{2 - x} = \frac{1}{1-x} - \frac{1}{2-x}$$
$$f^{n}(x) = \frac{n!}{(1-x)^{n+1}} - \frac{n!}{(2-x)^{n+1}}$$

\vspace{2mm}

Як обчислити похідну вищого порядку від параметрично заданої функції?

\vspace{3mm}

Ось \textit{приклад}:

$$\begin{cases} y = t^2 \\ x = t^3 + t\end{cases}$$

$$y'_x = \frac{y'_t (t)}{x'_t(t)} = \frac{(t^2)'}{(t^3 + t)'} = \frac{2t}{3t^2 + 1}$$

Початкова система задавала залежність $y$ від $x$. Тепер ми можемо побудувати систему залежності $y'$ від $x$:

$$\begin{cases} y_x'(t) = \frac{2t}{3t^2 + 1} \\ x = t^3 + t\end{cases}$$

$$y_x'' = (y_x')' = \frac{(y_x')_t'}{x_t'} = \frac{\Big( \frac{2t}{3t^2 + 1} \Big)'}{(t^3 + t)'} = \frac{-6t^2 + 2}{(3t^2 + 1)^3}$$

Тепер ми можемо побудувати залежність $y_x''$ від $x$:

$$\begin{cases} y_x''(t) =  \frac{-6t^2 + 2}{(3t^2 + 1)^3} \\ x = t^3 + t\end{cases}$$

І так далі...

\begin{center}
\textbf{\large Диференціал функції}
\end{center}

Функція $f$ називається диференційованою в точці $x_0$, якщо:

$$\exists \lim_{x \to x_0}\frac{f(x) - f(x_0)}{x - x_0}$$

Нехай $x - x_0 = \Delta x$. Тоді:

$$\exists \lim_{x \to x_0}\frac{f(x) - f(x-0)}{x - x_0} = \lim_{\Delta x \to 0} \frac{f(x_0 + \Delta x) - f(x_0)}{\Delta x} = f'(x)$$

$$\frac{f(x_0 + \Delta x) - f(x_0)}{\Delta x} = f'(x_0) + o(1)$$

Домножимо все на $\Delta x$:

$$\frac{f(x_0 + \Delta x) - f(x_0)} = \Delta x \cdot f'(x_0) + o(\Delta x)$$

Отже, $\Delta x f'(x_0)$ --- головна частина приросту функції.

\textit{Означення.} Лінійне відображення 
$$L(h) = f'(x_0) \cdot h$$
Називається диференціалом функції $f$ в точці $x_0$. Це позначається так:
$$d_{x_0}f(h) = f'(x_0) \cdot h$$

\textit{Приклад:}

$$f(x) = x^3,\ x_0 = 1$$
$$f'(x_0) = 3$$
$$d_{x_0}f(h) = 3h$$

\textit{Геометричний зміст диференціалу.} Диференціал задає рівняння прямої, яка паральна дотичній до графіку функції у точці $x_0$ і проходить через центр координат.

\vspace{5mm}

Розглянемо $f(x) = x$. Тоді:
$$f'(x_0) = 1,\ \textrm{для } \forall x_0 \in \mathbb{R}$$
Тоді:
$$d_{x_0}x(h) = 1 \cdot h = h$$

А тепер нехай $g$ --- довільна диференційована в $x_0$ функція:

$$(d_{x_0}g)(h) = g'(x_0) \cdot h = g'(x_0) \cdot d_{x_0} x(h)$$

Якщо ми приберемо $h$, отримуємо:

$$d_{x_0} g = g'(x_0) \cdot d_{x_0} x$$

Запис, який ще часто можна зустріти:

$$dg = g'(x_0)dx$$ 









\end{enumerate}

\end{document} 
