\documentclass[12pt]{report}
\usepackage[utf8x]{inputenc}
\usepackage[russian]{babel}
\usepackage{amssymb}
\usepackage{amsmath}
\usepackage[makeroom]{cancel}

\title{Конспекти лекцій з математичного аналізу Анікушина А.В. Модуль 3.}
\author{Автор текста @bezkorstanislav \\ Если есть ошибки, пишите ему в телеграм \\  \small{Aфтар выражает благодарность @vic778 за многочисленные поправки}}

\date{October 2019}

\begin{document}

\maketitle

\begin{center}
\textbf{\LargeДиференціальне числення функції однієї змінної}
\end{center}

\begin{center}
\textbf{\large{Означення похідної. Основні правила диференційонування}}
\end{center}

Нехай $f : \mathbb{R} \to \mathbb{R}$ --- деяка функція однієї змінної, $x_0 \in D_f,\ x_0 \in (D_f)^{'}$ .

\textbf{Означення.} Якщо $\exists \lim\limits_{x \to x_0} \frac{f(x) - f(x_0)}{x - x_0}$, то функція $f$ називається 
\textbf{диференційованою} в точці $x_0$, а сама границя називається \textbf{похідною} в точці $x_0$. І позначається $f'(x)$ або $\frac{d f(x)}{d x}$.

\vspace{3mm}

\textit{Зауваження 1.} Для функції однієї змінної ми ототожнили диференційованість та існування похідної.

\textit{Зауваження 2.} $$x - x_0 = \Delta x$$
$$f'(x_0) = \lim_{\Delta x \to 0} \frac{f(x_0 + \Delta x) - f(x_0)}{\Delta x}$$
$$f(x_0 + \Delta x) - f(x_0) = \Delta f(x)$$
Отже, похідна дорівнює відношенню зміни приросту функції до приросту аргументу, що породжує цей приріст.

\textit{Зауваження 3.} 
$$f'(x_0) = \lim_{x \to x_0} \frac{f(x) - f(x_0)}{x - x_0} \Longleftrightarrow \frac{f(x) - f(x_0)}{x - x_0} = f'(x_0) + o(1)$$
$$f(x) - f(x_0) = f'(x_0)(x-x_0) + o(x-x_0)$$
Отже, якщо має місце рівність:
$$f(x) - f(x_0) = A (x - x_0) + o(x - x_0) \Longrightarrow \exists f'(x_0) = A$$

\textbf{Теорема. (Необхідна умова диференційованості).} Функція $f$ є диференційованою в $x_0$ тільки тоді, коли $f$ --- неперервна в точці $x_0$.

\textbf{Доведення.} Щоб існувала похідна треба, щоб
$$f(x) - f(x_0) \to 0,\to x \to x_0$$.
З цього випливає, що $f$ неперервна в точці $x_0$.

\textbf{Теорема. (Диференційованість композиції функцій).}  Нехай дано функції $f$ і $g$. точка $x_0 \in D_{f \circ g},\ x_0 \in (D_{f \circ g})'$.

Якщо $g$ диференційована в точці $x_0$, а $f$ диференційована в точці $y_0 = g(x_0)$, то $f \circ g$ диференційована в точці $x_0$ і має місце рівність:
$$(f \circ g)' (x_0) = f'(g(y_0)) g'(x_0)$$

\textbf{Доведення.} 
$$(f \circ g) (x) - (f \circ g) (x_0) = f(g(x)) - f(g(x_0)) = f(y) - f(y_0) =$$ 
$$= f'(y_0)(y - y_0) + o(y - y_0) = f'(y_0)(g(x) - g(x_0)) + o(y - y_0) = $$
$$= f'(y_0)(g'(x_0)(x - x_0) + o(x - x_0)) + o(y - y_0) = $$
$$= f'(y_0)(g'(x_0)(x - x_0)(1 + o(1))) + o(g(x) - g(x_0)) = $$
$$= f'(y_0)(g'(x_0)(x - x_0)(1 + o(1))) + o(g(x_0)(x - x_0) + o(x - x_0))$$

Отже,

$$\lim_{x \to x_0} \frac{(f \circ g)(x) - (f \circ g)(x_0)}{x - x_0} = $$
$$=\lim_{x \to x_0}\frac{f'(y_0)(g'(x_0)(x - x_0)(1 + o(1))) + o(g(x_0)(x - x_0) + o(x - x_0))}{x - x_0} = $$
$$=\lim_{x \to x_0}f'(y_0)(g'(x_0)(1 + o(1))) + o(g(x_0) + o(1)) = $$
$$=\lim_{x \to x_0}f'(y_0)g'(x_0)(1 + o(1)) + o(1) = f'(y_0)g'(x_0)$$

\textbf{Теорема. (Лінійність похідної).} Нехай $f$ і $g$ --- диференційовані в точці $x_0$, $\alpha,\beta \in \mathbb{R}$, то:
$$(\alpha f + \beta g)' (x_0) = \alpha f'(x_0) + \beta g'(x_0)$$

\textbf{Доведення.} 

$$\lim_{x \to x_0} \frac{(\alpha f + \beta g)(x) - (\alpha f + \beta g)(x_0)}{x - x_0} = $$
$$\lim_{x \to x_0} \frac{ \alpha f(x) - \alpha f(x_0)}{x - x_0} + \frac{\beta g(x) - \beta g (x_0)}{x - x_0} = $$
$$\lim_{x \to x_0} \alpha \frac{ f(x) - f(x_0)}{x - x_0} + \lim_{x \to x_0}\beta \frac{g(x) - g (x_0)}{x - x_0}  = \alpha f'(x_0) + \beta g'(x_0)$$

\textbf{Теорема. Похідна добутку.} Нехай $f$ і $g$ диференційовані в точці $x_0$. 

Тоді функція $f \cdot g$ теж диференційована в точці $x_0$, при чому справедливе наступне співвідношення:
$$(f \circ g)' (x_0) = f'(x_0)g(x_0) + f(x_0) g'(x_0)$$

\textbf{Доведення.} За означенням:

$$\lim_{x \to x_0} \frac{f(x)g(x) - f(x_0)g(x_0)}{x - x_0} = $$
$$= \lim_{x \to x_0} \frac{f(x)g(x) - f(x)g(x_0) + f(x)g(x_0) - f(x_0)g(x_0)}{x - x_0} = $$
$$= \lim_{x \to x_0} \frac{f(x)(g(x) - g(x_0))}{x - x_0} + \frac{g(x_0)(f(x) - f(x_0))}{x - x_0} = $$

Оскільки $f$ --- диференційована в точці $x_0$, то $\exists \lim\limits_{x \to x_0} \frac{f(x) - f(x_0)}{x - x_0} = f'(x_0)$.

Абсолютно аналогічно $\exists \lim\limits_{x \to x_0} \frac{g(x) - g(x_0)}{x - x_0} = g'(x_0)$.

Окрім того $f$ --- диференційована в точці $x_0$, а значить $f$ --- неперервна в точці $x_0$, а значить:
$$\lim_{x \to x_0} f(x) = f(x_0)$$

Отже:

$$\lim_{x \to x_0} \frac{f(x)(g(x) - g(x_0))}{x - x_0} + \frac{g(x_0)(f(x) - f(x_0))}{x - x_0} = $$
$$= f(x_0) g'(x_0) + g(x_0) f'(x_0)$$

\textbf{Теорема. Похідна частки.} Нехай $f$ і $g$ диференційовані в точці $x_0$ і $g(x_0) \neq 0$.

Тоді, $\frac{f}{g}$ теж диференційована в точці $x_0$ і $\Big(\frac{f}{g} \Big)'(x_0) = \frac{f'(x_0) g(x_0) - f(x_0)g'(x_0)}{g^2(x_0)}$

\textbf{Доведення.} Аналогічно. Це доведення на дз :).

\textit{Приклад.} Обчислити похідну від $f$ від функції $f$ в точці $x_0$.
\begin{enumerate}

\item $f(x) = x^3$.
$$f'(x_0) = \lim_{x\to x_0}\frac{f(x) - f(x_0)}{x - x_0} = \lim_{x\to x_0}\frac{x^3 - x_0^3}{x - x_0} = \ldots = 3 x_0^2$$

\item $f(x) = \begin{cases} x^3,& x \neq 2 \\ 8, & x = 2\end{cases}$
$$f'(x_0) = \lim_{x \to x_0}\frac{f(x) - f(x_0)}{x - x_0}$$
\begin{enumerate}
\item $x_0 = 2$. $f'(x_0) = \frac{x^3 - 8}{x - 2} = 3 \cdot 4 = 12.$
\item $x_0 \neq 2$. $\lim\limits_{x \to x_0} \frac{x^3 - x_0^3}{x - x_0} = 3 x_0^2$. 
\end{enumerate}

\item $f(x) = \begin{cases} x^3,& x \neq 2 \\ 7, & x = 2\end{cases}$
$$f'(x_0) = \lim_{x \to x_0}\frac{f(x) - f(x_0)}{x - x_0}$$
\begin{enumerate}
\item $x_0 \neq 2$. $\lim\limits_{x \to x_0} \frac{x^3 - x_0^3}{x - x_0} = 3 x_0^2$. 
\item $x_0 = 2$. $f'(x_0) = \lim\limits_{x\to 2}\frac{x^3 - 7}{x - 2} = \lim\limits_{x\to 2}\frac{8 - 7}{2 - 2} = \lim\limits_{x\to 2}\frac{1}{0} = +\infty$.

Похідна не може дорівнювати нескінченності, отже $\nexists f'(2)$. 
\end{enumerate}

\item $f(x) = \begin{cases} x^3,& x \geq 2 \\ 12x - 16, & x < 2\end{cases}$

Як \textbf{не треба} робити:
$$f'(x) = \begin{cases} 3 x^2,& x \geq 2 \\ 12, & x < 2\end{cases}$$
Як \textbf{треба} робити:

\vspace{1mm}

Розглядаємо $3$ випадки:

\begin{enumerate}
\item $x_0 > 2$.

$$\lim_{x \to x_0}\frac{f(x) - f(x_0)}{x - x_0} = \lim_{x \to x_0}\frac{x^3 - x_0^3}{x - x_0} = 2 x_0^2$$

\item $x_0 < 2$.

$$\lim_{x \to x_0}\frac{f(x) - f(x_0)}{x - x_0} = 12$$

\item $x = 2$.

$$\lim_{x \to x_0}\frac{f(x) - f(2)}{x - 2} = \lim_{x \to 2+0}\frac{f(x) - f(2)}{x - 2} = \lim_{x \to 2-0}\frac{f(x) - f(2)}{x - 2} = 12$$
\end{enumerate}

Отже, функція диференційована на всіх дійсній прямій.
 
\end{enumerate}

\textbf{Теорема. (Про диференційованість оберненої функції)}. Формулювання і доведення на додому.

\textit{Зауваженн.} Функція $f$ називається диференційованою на множині $A$ якщо $f$ має 
похідну в кожній точці $x_0 \in A$. Це позначається як $f \in D(A)$, де $D(A)$ --- множина диференційованих на $A$ функцій. 

\begin{center}
\textbf{\largeОдносторонні похідні}
\end{center}

Нехай $x_0 \in (D_f \cap (-\infty, x_0))' \cap (D_f \cap (x_0, +\infty))'\cap D_f$. Це рівносильно тому, що 
$$\exists \{ x_n\}_{n=1}^{\infty} \in D_f : x_n < x_0,\ x_n \to x_0, n \to \infty $$
Границя $\lim\limits_{x\to x_0-}\frac{f(x) - f(x_0)}{x - x_0}$ називається лівосторонньою похідною $f$ в точці $x_0$. Позначаємо її як $f'_{л}(x)$.

Аналогічно, $f'_{п} (x_0) = \lim\limits_{x \to x_0+} \frac{f(x) - f(x_0)}{x - x_0}$

\vspace{3mm}

\textit{Приклади:}

\begin{enumerate}

\item $f(x) = \begin{cases} 0, & x \leq 0 \\ 1, & x > 1\end{cases}$
$$f'_{л}(x_0) = \lim_{x \to x_0} \frac{0 - 0}{x - 0} = 0$$
$$f'_{п}(x_0) = \lim_{x \to x_0} \frac{1 - 0}{x - 0} = +\infty \Longrightarrow \nexists f'_{n}(x_0)$$

\item $f(x) = \begin{cases} 0, & x \leq 0 \\ x, & x > 1\end{cases}$
$$f'_{л}(x_0) = \lim_{x \to x_0} \frac{0 - 0}{x - 0} = 0$$
$$f'_{п}(x_0) = \lim_{x \to x_0} \frac{x - 0}{x - 0} = 1$$
\end{enumerate}

\textbf{Теорема. (Критерій диференційованості}.) 

Нехай $x_0 \in (D_f \cap (-\infty, x_0))' \cap (D_f \cap (x_0, +\infty))' \cap x_0 \in D_f$. Тоді:
$$\exists f'(x_0) \Longleftrightarrow \exists f'_{л} \exists f'_{п}, f'_{л} = f'_{п}$$

\textbf{Доведення.} 
$$\exists f'(x_0) \Longrightarrow \lim_{x \to x_0}\frac{f(x) - f(x_0)}{x - x_0} \Longleftrightarrow$$
$$\Longleftrightarrow \exists \lim_{x\to x_0+}\frac{f(x) - f(x_0)}{x - x_0},\ \exists \lim_{x\to x_0-}\frac{f(x) - f(x_0)}{x - x_0}, \lim_{x\to x_0+}\frac{f(x) - f(x_0)}{x - x_0} = \lim_{x\to x_0-}\frac{f(x) - f(x_0)}{x - x_0}$$
$$\Longleftrightarrow \exists f'_{п}(x_0),\ \exists f'_{л}(x_0),\ f'_{п}(x_0) = f'_{л}(x_0)$$


\textit{Зауваження.}
$$f'_{л} \neq \lim_{x\to x_0-}f'(x)$$



\end{document} 
